\input{../common.tex}
\renewcommand{\arraystretch}{1.5}

\newcommand{\headertext}{EE240B HW 2}
\renewcommand{\thesubsection}{\thesection.\alph{subsection}}

\newcommand{\Ohms}{\text{ }\Omega}
\newcommand{\MOhms}{\text{ M}\Omega}

\newcommand{\VNoise}{\frac{\text{V}}{\sqrt{\text{Hz}}}}
\newcommand{\nVNoise}{\frac{\text{nV}}{\sqrt{\text{Hz}}}}

\title{\vspace{-0.4in}\Large \bf \headertext \vspace{-0.1in}}
\author{Vighnesh Iyer}

\date{\today}
\maketitle

\markboth{\headertext}{\headertext}
\thispagestyle{empty}

\section{Electronic Noise}
\begin{figure}[H]
  \centering
  \includegraphics[width=0.5\textwidth]{figs/problem1.png}
\end{figure}

\begin{enumerate}[label=(\alph*)]
  \item {\color{blue}Calculate the input referred noise in $V / \sqrt{\text{Hz}}$ achieved with the two amplifiers for $R_S = 50 \Omega, 5 \text{M}\Omega$. Only consider the amplifier noise. Assume the amplifier voltage and current sources are uncorrelated.}

  \begin{center}
    \begin{tabular}{ |c|c|c| }
      \hline
      $R_s$ & Amplifier A & Amplifier B \\ \hline
      $50 \Ohms$ & $1.05 \nVNoise$ & $10 \nVNoise$ \\ \hline
      $5 \MOhms$ & $5000 \nVNoise$ & $15 \nVNoise$ \\ \hline
    \end{tabular}
  \end{center}

  \item {\color{blue}Significance of the result?}

  If the input voltage source's resistance is low, an amplifier with low voltage noise should be chosen (since the amplifier's current noise won't see a large resistance). Use a BJT op-amp.

  If the input voltage source's resistance is high, an amplifier with low current noise should be chosen. This is usually a FET op-amp.
\end{enumerate}

\section{Amplifier Noise}
Derive analytical expressions as a function of $g_m, \gamma, f_T, R_L, R_S,$ and $C_L$. Consider only thermal noise from $R_S, R_L$, and transistor shot noise. Ignore flicker noise, $r_o$, and body effect. $C_L$ is fixed and $C_{gs}$ is expressed in terms of $f_T$.
\begin{figure}[H]
  \centering
  \includegraphics[width=0.3\textwidth]{figs/problem2.png}
\end{figure}

\begin{enumerate}[label=(\alph*)]
  \item {\color{blue}The voltage gain $a_v(s) = v_o / v_i$}

    This is a standard source-degenerated common-source amplifier with a complex load. We need to consider $C_{gs}$ which will give a 2nd-order transfer function and makes the algebra complicated. I'm using the sympy CAS to do the algebra. I'll provide the source and the derived solution.

    \begin{minted}{python}
from sympy import *
def ll(a, b): # Put impedances a and b in parallel symbolically
    return (a*b) / (a + b)
s, RL, CL, ZL, Cgs, Rs, gm = symbols("s R_L C_L Z_L C_{gs} R_s g_m")
igate, isource, idrain = symbols("i_g i_s i_d")
vs, vd, vi = symbols("v_s v_d v_i")
sol = nonlinsolve([
    igate - ((vi - vs) / (1 / (s*Cgs))),
    isource - (vs / Rs),
    idrain - gm*(vi - vs),
    -vd - idrain*ZL,
    isource - igate - idrain
], [igate, isource, idrain, vd, vs])
vd_sol = sol.args[0][3]
gain = vd_sol / vi
ZL_sol = ll(CL, RL)
ZL_sol = ZL_sol.subs(CL, (1 / (s*CL))).simplify()
display(gain.subs(ZL, ZL_sol).simplify())
    \end{minted}
    \begin{align*}
      a_v(s) = - \frac{R_{L} g_{m}}{\left(C_{L} R_{L} s + 1\right) \left(C_{gs} R_{s} s + R_{s} g_{m} + 1\right)}
    \end{align*}

  \item {\color{blue}The PSD of the noise in $\VNoise$ at node $v_o$}

    I derived the noise due to $R_S, R_L, $ and the transistor.
    \begin{align*}
      V_{out, R_s} &= \frac{R_{L} g_{m} v_{nS}}{\left(C_{L} R_{L} s + 1\right) \left(C_{gs} R_{s} s + R_{s} g_{m} + 1\right)} \\
      V_{out, R_L} &= \frac{R_{L} i_{nL}}{C_{L} R_{L} s + 1}
    \end{align*}
  \item {\color{blue}The total noise integrated over all frequencies at node $v_o$}
  \item {\color{blue}The dynamic range of the circuit as a function of the peak-to-peak voltage range $V_{o,pp}$ at node $v_o$}
  \item {\color{blue}The minimum detectable signal in $\nVNoise$ at low frequency at $\mu V_{rms}$ based on the total noise at the output of the amplifier}
  \item {\color{blue}Comment on the effect of $R_S$ on the dynamic range and minimum detectable signal of the circuit}
\end{enumerate}
\end{document}
